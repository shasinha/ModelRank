\documentclass[twoside,11pt]{article}

% Any additional packages needed should be included after jmlr2e.
% Note that jmlr2e.sty includes epsfig, amssymb, natbib and graphicx,
% and defines many common macros, such as 'proof' and 'example'.
%
% It also sets the bibliographystyle to plainnat; for more information on
% natbib citation styles, see the natbib documentation, a copy of which
% is archived at http://www.jmlr.org/format/natbib.pdf

\usepackage{jmlr2e}

% Definitions of handy macros can go here

\newcommand{\dataset}{{\cal D}}
\newcommand{\fracpartial}[2]{\frac{\partial #1}{\partial  #2}}

% Heading arguments are {volume}{year}{pages}{submitted}{published}{author-full-names}

\jmlrheading{1}{2021}{1-48}{4/00}{10/00}{Shashvat Sinha and Peter L. Russo}

% Short headings should be running head and authors last names

\ShortHeadings{Model Rank for Networks of Models}{Sinha and Russo}
\firstpageno{1}

\begin{document}

\title{Model Rank for Networks of Models}

\author{\name Shashvat Sinha \email shashvat@sinha.com \\
       \addr Morris Plains\\
       NJ 07950, USA
       \AND
       \name Peter L.\ Russo \email plrusso@gmail.com \\
       \addr Jersey City\\
       NJ 07310, USA}

\editor{Jon Hill}

\maketitle

\begin{abstract}%   <- trailing '%' for backward compatibility of .sty file
This paper describes a method to rank risk models that comprise a network of models in order to determine the models with the highest risk ranking. This paper draws on the work done by \citet{brin_page} in the PageRank algorithm. 
\end{abstract}

\begin{keywords}
  Model Risk, PageRank, Model Networks
\end{keywords}

\section{Introduction}

{\noindent \em Describe Model Risk:}\\
Model Risk has been a principal or key risk in financial institutions since the introduction of the Dodd-Frank act and the associated Fed supervisory letter SR-11-7.\\

{\noindent \em Describe the space we are operating in - large model inventories of interconnected models:}\\
Financial institutions have large inventories of models. Models are not just used individually, they are often used collectively, where the outputs of one model are the inputs to other models. Such relationships can be considered networks of models ....\\


\section{Pinpointing Risk}

{\noindent \em Go into further detail about the problem - pinpointing which models create the most risk and marshalling resources around validating and monitoring them, and also their improvement/redevelopment/evolution:}\\
Lorem ipsum.\\

\subsection{Validation}
Lorem ipsum.\\

\subsection{Monitoring}
Lorem ipsum.\\

\subsection{Redevelopment}
Lorem ipsum.\\


\section{Parallel Problems}
{\noindent \em Describe how similar issues exist elsewhere, therefore our work is derivative of them:}\\
Lorem Ipsum.\\


\section{Page Rank}
{\noindent \em Describe the PageRank algorithm:}\\
Lorem Ipsum.\\

\subsection{Overview}
{\noindent \em What problem does it solve, and how does it solve it.}\\
Lorem ipsum.\\

\subsection{Expanded Applications}
{\noindent \em Describe how PageRank has been applied to other areas, especially if those areas are mentioned in the previous section.}\\
Lorem ipsum.\\

\section{Application to Model Risk}
{\noindent \em Describe how PageRank can similarly be applied to the problem in Model Risk}\\
Lorem ipsum.\\

\subsection{Expanded Applications}
{\noindent \em Describe how PageRank has been applied to other areas, especially if those areas are mentioned in the previous section.}\\
Lorem ipsum.\\

\subsection{Model Rank Approach}
{\noindent \em Describe the model rank approach will work}\\
Lorem ipsum.\\


\section{Model Rank in Detail}

\subsection{Setup}
{\noindent \em Describe the environment - how was model inventory and network arranged and organized. Also mention how others could set up their model inventories similarly}\\
Lorem ipsum.\\

\subsection{Implementation}
{\noindent \em Describe the code, how it works. Relate it to PageRank and similar algorithms, and then relate it to the data setup from the previous subsection. Code itself can be shown in a appendix, or fragments shown there and full code will be in Github, with a reference here}\\
Lorem ipsum.\\


\subsection{Results and Demonstration}
{\noindent \em Show how it worked, what results were found. Some diagrams and charts, especially that show an evolution of the iterative results}\\
Lorem ipsum.\\


\section{Conclusion}

\subsection{Observations}
{\noindent \em What did we see, what were our findings}\\
Lorem ipsum.\\

\subsection{Gaps}
{\noindent \em What did we expect to find, but did not}\\
Lorem ipsum.\\

\subsection{Application}
{\noindent \em How can the readers of this paper benefit from it in their organisations, whats in it for them}\\
Lorem ipsum.\\

\subsection{Next}
{\noindent \em What can be done further, will form the subject of other papers}\\
Lorem ipsum.\\

% Acknowledgements should go at the end, before appendices and references

\acks{Thank our employers if we decide to take their help or implement/prototype it at work.}

% Manual newpage inserted to improve layout of sample file - not
% needed in general before appendices/bibliography.

\newpage

\appendix
\section*{Appendix A.}
\label{app:theorem}

% Note: in this sample, the section number is hard-coded in. Following
% proper LaTeX conventions, it should properly be coded as a reference:

%In this appendix we prove the following theorem from
%Section~\ref{sec:textree-generalization}:

In this appendix we prove the following theorem from
Section~6.2:

\noindent
{\bf Theorem} {\it Let $u,v,w$ be discrete variables such that $v, w$ do
not co-occur with $u$ (i.e., $u\neq0\;\Rightarrow \;v=w=0$ in a given
dataset $\dataset$). Let $N_{v0},N_{w0}$ be the number of data points for
which $v=0, w=0$ respectively, and let $I_{uv},I_{uw}$ be the
respective empirical mutual information values based on the sample
$\dataset$. Then
\[
	N_{v0} \;>\; N_{w0}\;\;\Rightarrow\;\;I_{uv} \;\leq\;I_{uw}
\]
with equality only if $u$ is identically 0.} \hfill\BlackBox

\noindent
{\bf Proof}. We use the notation:
\[
P_v(i) \;=\;\frac{N_v^i}{N},\;\;\;i \neq 0;\;\;\;
P_{v0}\;\equiv\;P_v(0)\; = \;1 - \sum_{i\neq 0}P_v(i).
\]
These values represent the (empirical) probabilities of $v$
taking value $i\neq 0$ and 0 respectively.  Entropies will be denoted
by $H$. We aim to show that $\fracpartial{I_{uv}}{P_{v0}} < 0$....\\

{\noindent \em Remainder omitted in this sample. See http://www.jmlr.org/papers/ for full paper.}


\vskip 0.2in
\bibliography{modelrank}

\end{document}